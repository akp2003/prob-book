% In this file you should put all LaTeX macros to be used
% both by the pdf version and the web version.
% This should be most of your macros.

%modified prob.tex

\usepackage[dvips]{epsfig}	% specify `dvips' driver
%\usepackage[pdftex]{epsfig}	% specify `pdftex' driver
\usepackage{makeidx}
\setlength{\textheight}{8.1in}
\setlength{\textwidth}{5in}
\setlength{\topmargin}{.25in}
\setlength{\oddsidemargin}{.75in} % 1.25 for book Change to .75 for web version
\setlength{\evensidemargin}{.75in} % .25 for book Change to .75 for web version

%
% Low-level figure positioning.  Change as appropriate to your system.
%
\newcommand{\barefig}[2]{\makebox{\includegraphics*[width=#2] {figures/#1.ps}}} 
%
% Higher-level figure commands.
%
\newcommand{\NA}{{\rm NA}}
\usepackage{latexsym}
\newif\ifdiscrete
%\discretetrue
\discretefalse
\newcommand{\choice}[2]{\ifdiscrete{#1}\else{#2}\fi}

%\putfig{3.5truein}{PSfig1.3}{Peter's winnings in 40 plays of heads or tails.}{fig 1.3}
\newcommand{\putfig}[4]
{\begin{figure}
		\centerline{\barefig{#2}{#1}}
		\caption{#3}
		\label{#4}
\end{figure}}

\newcommand{\putfigdouble}[6]
{\begin{figure}
		\centerline{\barefig{#2}{#1}}
		\centerline{\barefig{#4}{#3}}
		\caption{#5}
		\label{#6}
\end{figure}}

\newcommand{\nocaption}[4]
{\begin{figure}
		\centerline{\barefig{#2}{#1}}
		%\caption{#3}
		\label{#4}
\end{figure}}

\newcommand{\exercises}{\subsection*{Exercises}}
%\newcommand{\emx}[1]{{\emx{#1}\/}}
%\newcommand{\mat}[1]{{\mbox{\bf#1}\/}}
\newcommand{\emx}[1]{{\em{#1}\/}}
\newcommand{\mat}[1]{{\mbox{\bf#1}\/}}


\newenvironment{scope}{}{}
\newenvironment{shrink}{\begin{scope} \small}{\end{scope}}
\newenvironment{progout}{\begin{shrink}}{\end{shrink}}


\newcount\exerciseno
\def\i{\advance\exerciseno by 1\bigskip
	\item[\bf{\the\exerciseno}]}
\def\istar{\advance\exerciseno by 1\bigskip
	\item[\bf{*\the\exerciseno}]}

\newenvironment{LJSItem}{
	
	\newcommand{\exref}[1]{\ref{##1}}
	\def\i{\renewcommand{\labelenumi}{\bf\theenumi}\item}
	
	\def\istar{\renewcommand{\labelenumi}{\bf*\theenumi}\item}
	
	\begin{enumerate}}
	{\end{enumerate}}

\def\partone{ch1,ch2,ch3,ch4,ch5,ch6}
\def\parttwo{ch7,ch8,ch9,ch10,ch11,ch12}
\def\all{front,\partone,\parttwo,back}
\newcommand{\secstoprocess}{\all}
%\typein[\secstoprocess]{enter the sections to process: }
\includeonly{\secstoprocess}

\makeindex